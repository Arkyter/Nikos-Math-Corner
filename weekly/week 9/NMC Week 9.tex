\documentclass[11pt]{scrartcl}
\usepackage[sexy]{{style_files/evan}}

\usepackage{{style_files/NMC}}
\usepackage{standalone}
\usepackage{import}

\begin{document}
\title{NMC Problem Set \#9} % add # of pset
\date{Oct. 16, 2022} % add date
\maketitle

\section*{Welcome!}

This is a selection of interesting problems derived from curious thoughts, curated so you can nibble on them throughout the week! The point of this document is to introduce you to fun puzzles that require thinking. We recommend you try the ones that you find interesting! Feel free to work on them with others (even us teachers!). Harder problems are marked with chilies (\fullchili), in case you want to challenge yourself.
\newline\newline
Have fun! \textit{Note: New variants on these problems may be released throughout the week. Remember to check back once in a while!}
    
\section{Algebra}
\begin{enumerate}[label=\textbf{A\arabic*}.]
    \item \textbf{Reprise of Fibonacci} \newline
    Define $F_{n}$ as the $n$th Fibonacci number, with the sequence starting at $F_1 = 1$, $F_2 = 1$, $F_3 = 2$... Show that
    \[ F_{n+1} = \sum_{k=0}^n \binom{n-k}{k}. \]
    
    \item \textbf{Machin-like Formulas} \newline
    The \textit{Machin-like formula} is defined as any equation of the form,
    \[ k\frac{\pi}{4} = \sum c_n \arctan \frac{a_n}{b_n}, \]
    with positive integer $k$ and non-negative integers $(a_n), (b_n), (c_n)$. The classic such formula is as follows:
    \[ \frac{\pi}{4} = 4\arctan\frac{1}{5} - \arctan\frac{1}{239}. \]
    
    \begin{enumerate}
        \item Prove that the classic Machin-like formula holds true.
        
        \item Determine $k$ in the following equation,
        \[ \frac{\pi}{4} = 2 \arctan \frac{1}{3} + \arctan k.  \]
        
        \item Create your own!
    \end{enumerate}
    
    
\end{enumerate}

\newpage
\section{Combinatorics}
\begin{enumerate}[label=\textbf{C\arabic*}.]
    \item \textbf{Alternating Sequences} \newline
    Call a permutation $(a_1, a_2, a_3, \dots, a_n)$ of $\{1, 2, 3, \dots, n\}$ \textit{alternating} if
    \[ a_1 > a_2 < a_3 > a_4 < a_5 > \dots. \]
    Define $E_n$ as the number of alternating permutations of $\{1, 2, 3, \dots, n\}$.
    
    \begin{enumerate}
        \item (\fullchili) We see that $E_0 = E_1 = 1$. Create a combinatorial argument that
        \[ 2E_{n+1} = \sum_{k=0}^n \binom{n}{k} E_k E_{n-k} \]
        for $n \geq 1$.
        
        \item (\fullchili \hspace{1pt} $\times$ 2 \footnote{calculus is handy here}) Show that
        \[ \sum_{n \geq 0} E_n \frac{x^n}{n!} = \sec{x} + \tan{x}. \]
    \end{enumerate}
\end{enumerate}

\newpage
\section{Geometry}
\begin{enumerate}[label=\textbf{G\arabic*}.]
    \item \textbf{The Kakeya Needle Problem} \newline
    This problem asks whether there is a minimum area $D$ in which a needle of unit length can be rotated $360$ degrees. 
    
    \begin{enumerate}
        \item Suppose we take a circle with diameter $1$. Naturally, we can fully rotate a needle of unit length inside. What's the minimum area of an equilateral triangle that allows for a needle to fully rotate?
        
        \item (\halfchili) Is the shape above the minimal achievable area? If so, provide an example of a shape that allows for the needle to rotate fully with less area than our equilateral triangle above.
    \end{enumerate}
    
\end{enumerate}

\newpage
\section{Number Theory}
\begin{enumerate}[label=\textbf{N\arabic*}.]
    \item \textbf{Arithmetic functions} \newline
    Arithmetic functions refer to any function that takes natural numbers as inputs and outputs a subset of complex numbers (which includes the reals). Below, we list some identities on \emph{additive} and \emph{multiplicative} arithmetic functions.
    \begin{enumerate}
        \item Given a prime $p$, we can define a function $\nu_p(n)$ that tells us the exponent of $p$ in the factorization of $n$. For example, $\nu_3(18) = 2$ because $18 = 2\cdot 3^2$. \newline
        Convince yourself that $\nu_p$ is additive, that is, for all numbers $a,b$ this equation holds:
        \[ \nu_p(ab) = \nu_p(a)+\nu_p(b) \]
        
        \item With $d(n)$, we denote the number of divisors of $n$. For example, $d(12) = 6$ because there are $6$ numbers that divide $12$: $1,2,3,4,6,12$. \newline
        Show that the function $d$ is multiplicative. This means that if $m, n$ are co-prime integers, then
        \[ d(mn) = d(m)d(n). \]
        
        \item This is Euler's totient function, $\varphi(n)$. It calculates how many numbers from $1, ..., n$ are co-prime with $n$. Show that $\varphi$ is multiplicative.
        
        \item (\fullchili) Suppose $f$ is a multiplicative function. If $f^*(n) = f(1) + \cdots + f(n)$, ranging over the divisors $d$ of $n$, show that $f^*$ is also multiplicative.
    \end{enumerate}
\end{enumerate}

\end{document}
